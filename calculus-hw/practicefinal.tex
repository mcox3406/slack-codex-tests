\documentclass[12pt]{article}
\usepackage[utf8]{inputenc}
\usepackage{amsmath}
\usepackage{amssymb}
\usepackage{amsfonts}
\usepackage[margin=1in]{geometry}
\usepackage{fancyhdr}
\usepackage{lastpage}
\usepackage{enumitem}
\usepackage{xcolor}

% --- Header and Footer Setup ---
\pagestyle{fancy}
\fancyhf{} % Clear all header and footer fields
\fancyhead[L]{Calculus I (UN1101)}
\fancyhead[R]{Practice Final A}
\fancyfoot[C]{Page \thepage\ of \pageref{LastPage}}
\renewcommand{\headrulewidth}{0.4pt}
\renewcommand{\footrulewidth}{0pt} % No line at the footer to match the source

% --- Document Start ---
\begin{document}

% --- Cover Page ---
\begin{center}
    \Large\textbf{Calculus I Practice Final A}
\end{center}
\vspace{1cm}

\begin{itemize}
    \item This exam contains \pageref{LastPage} pages (including this cover page) and 6 questions.
    \item The total number of possible points is 80 points. You will have 150 minutes to complete this exam.
    \item Print your name and UNI in the space above.
    \item Answer the questions in the space provided on the question sheets. You may use extra paper.
    \item Clearly identify and simplify your answers. You will not receive full credit if there are multiple apparent answers, even if one of them is correct.
    \item Write legibly and show your work. You may receive partial credit for intermediate steps. Correct answers without any reasoning or work will not receive full credit.
    \item No calculators, computational devices, or consulting other people during the duration of this exam. Any cheating will result in an automatic failing grade in the course and potential administrative action.
    \item You may consult your notes and textbook for this exam. This does not include WebAssign, Courseworks, or other online resources.
    \item Upload your exam to Gradescope via PDF or images at the end of the time allotted. At the end of the exam, you have 15 minutes to upload your exam. Any exams uploaded after the end of the upload period will not be accepted.
    \item Remain in the Zoom call with your camera. If you need to leave your work area for any reason, please inform the instructor beforehand.
\end{itemize}
\newpage
% --- Question 1 ---
\section*{1. Consider the function}
\[ f(x)=\begin{cases} \sin(x) & \text{if } x > \pi \\ x-\pi & \text{if } x \le \pi \end{cases} \]

\begin{enumerate}[label=(\alph*)]
    \item (4 points) Identify the real numbers at which $f(x)$ is discontinuous.
    \textit{Hint: You should justify why $f(x)$ is discontinuous at certain values of x and why $f(x)$ is continuous everywhere else.}

    {\color{blue}\textbf{Solution.} On the region $x>\pi$, $f(x)=\sin x$ is continuous. On $x\leq\pi$, $f(x)=x-\pi$ is a polynomial and is likewise continuous. The only potential issue is at $x=\pi$, where the rule switches. We compute
    \[
        \lim_{x\to \pi^-}f(x)=\lim_{x\to\pi^-}(x-\pi)=0,\qquad \lim_{x\to \pi^+}f(x)=\lim_{x\to\pi^+}\sin x=\sin\pi=0,
    \]
    and the actual value $f(\pi)=\pi-\pi=0$. Because the left- and right-hand limits agree with $f(\pi)$, the function is continuous at $x=\pi$. Therefore $f$ is continuous for every real number, so it has \textbf{no discontinuities}.}
    
    \item (4 points) Identify the horizontal and vertical asymptotes of $f(x)$.

    {\color{blue}\textbf{Solution.} For $x>\pi$ the function equals $\sin x$, which oscillates between $-1$ and $1$ forever without approaching a single value, so it has \emph{no horizontal asymptote} as $x\to\infty$. For $x\leq\pi$ the expression is $x-\pi$, a line whose magnitude grows without bound as $x\to -\infty$, so it also has no horizontal asymptote in that direction. Because each piece is defined and finite for every $x$ in its domain, $f$ never diverges to $\pm\infty$ at a finite $x$-value, hence there are \textbf{no vertical asymptotes either}.}

    \item (4 points) What does the Mean Value Theorem say about $f(x)$ on the interval $[2\pi, 3\pi]$?

    {\color{blue}\textbf{Solution.} On $[2\pi,3\pi]$ the function is $f(x)=\sin x$, which is continuous on the closed interval and differentiable on the open interval, so the hypotheses of the Mean Value Theorem (MVT) are satisfied. The MVT guarantees a number $c\in(2\pi,3\pi)$ such that
    \[
        f'(c)=\frac{f(3\pi)-f(2\pi)}{3\pi-2\pi}=\frac{\sin(3\pi)-\sin(2\pi)}{\pi}=0.
    \]
    Because $f'(x)=\cos x$, the equation $\cos c=0$ forces $c=\tfrac{5\pi}{2}$ (the unique solution in $(2\pi,3\pi)$). Thus the theorem tells us that the instantaneous rate of change vanishes at $c=\tfrac{5\pi}{2}$ within that interval.}
    
\end{enumerate}

\newpage

% --- Question 2 ---
\section*{2. Find the limit if it exists. If the limit does not exist, explain why.}

\begin{enumerate}[label=(\alph*)]
    \item (4 points)
    \[ \lim_{x\to 1} \frac{x^2+1}{x-1} \]
    {\color{blue}\textbf{Solution.} The numerator approaches $1^2+1=2$ while the denominator approaches $0$. Approaching from t
    he right, $x-1>0$ and the quotient grows without bound to $+\infty$; approaching from the left, $x-1<0$ and the quotient dive
    rges to $-\infty$. Because the one-sided limits disagree and are infinite in magnitude, the limit does \emph{not} exist (and
     $x=1$ is a vertical asymptote).}
    
    \item (4 points)
    \[ \lim_{x\to \infty} \frac{e^{-x}}{e^x+1} \]
    {\color{blue}\textbf{Solution.} Rewrite $e^{-x}=\dfrac{1}{e^x}$. Then
    \[
        \frac{e^{-x}}{e^x+1}=\frac{1}{e^x(e^x+1)}.
    \]
    As $x\to\infty$, the denominator tends to $\infty$, so the fraction tends to $0$. Thus the limit equals $\boxed{0}$.}
    
    \item (4 points)
    \[ \lim_{x\to \infty} \frac{2x^2-2}{-2x+\sin(x)} \]
    {\color{blue}\textbf{Solution.} Factor out $x$ from numerator and denominator:
    \[
        \frac{2x^2-2}{-2x+\sin x}=\frac{x\left(2x-\frac{2}{x}\right)}{x\left(-2+\frac{\sin x}{x}\right)}=\frac{2x-\frac{2}{x}}{-2+\frac{\sin x}{x}}.
    \]
    As $x\to\infty$, $\tfrac{2}{x}\to 0$ and $\tfrac{\sin x}{x}\to 0$, so the expression behaves like $\dfrac{2x}{-2}=-x$, wh
    ich diverges to $-\infty$. Therefore the limit does not exist as a finite number; instead the quotient decreases without bou
    nd and we can say it tends to $-\infty$.}
    
    \item (4 points)
    \[ \lim_{x\to 0} x^2 \cos\left(\frac{1}{x}\right) \]
    {\color{blue}\textbf{Solution.} The cosine factor always satisfies $-1\leq \cos(1/x)\leq 1$, so
    \[
        -x^2\leq x^2\cos\left(\frac{1}{x}\right)\leq x^2.
    \]
    Both bounding functions tend to $0$ as $x\to 0$, so by the Squeeze Theorem the original limit also equals $0$.}
    
\end{enumerate}

\newpage

% --- Question 3 ---
\section*{3. Consider the function}
\[ f(x) = 2x^3 + 9x^2 + 12x + 1. \]

\begin{enumerate}[label=(\alph*)]
    \item (3 points) State the domain and range of the function $f(x)$.
    {\color{blue}\textbf{Solution.} $f(x)$ is a polynomial, and polynomials are defined for every real input, so the domain is $(-\infty,\infty)$. Moreover, the leading term $2x^3$ dominates as $|x|\to\infty$, yielding $f(x)\to -\infty$ as $x\to -\infty$ and $f(x)\to \infty$ as $x\to \infty$. Because the function is continuous and attains arbitrarily large positive and negative values, its range is also $(-\infty,\infty)$.}

    \item (4 points) Find $f'(x)$ and $f''(x)$.
    {\color{blue}\textbf{Solution.} Differentiate term-by-term:
    \[
        f'(x)=6x^2+18x+12=6(x+1)(x+2),\qquad f''(x)=12x+18=6(2x+3).
    \]
    These derivatives will be useful for locating critical points and inflection points later.}
    
\end{enumerate}

\newpage

\begin{enumerate}[label=(\alph*)]\addtocounter{enumi}{2} % Continue numbering from previous page
    \item (5 points) Find the local extrema of $f(x)$.
    {\color{blue}\textbf{Solution.} Local extrema occur at critical points where $f'(x)=0$ or where $f'$ is undefined. Since $f'$
    is a polynomial, it is defined everywhere, so we solve $6(x+1)(x+2)=0$. This gives critical points at $x=-2$ and $x=-1$. Eva
    luating the second derivative, $f''(-2)=6(2(-2)+3)=6(-1)=-6<0$, so $f$ has a local \emph{maximum} at $x=-2$. The function val
    ue there is
    \[
        f(-2)=2(-2)^3+9(-2)^2+12(-2)+1=-16+36-24+1=-3.
    \]
    At $x=-1$, $f''(-1)=6(2(-1)+3)=6(1)=6>0$, so $f$ has a local \emph{minimum}. The value is
    \[
        f(-1)=2(-1)^3+9(-1)^2+12(-1)+1=-2+9-12+1=-4.
    \]
    Thus the local maximum is at $(-2,-3)$ and the local minimum is at $(-1,-4)$.}
    
    \item (4 points) Find all of the values of $x$ where $f(x)$ has an inflection point.
    {\color{blue}\textbf{Solution.} Inflection points occur where $f''(x)$ changes sign, which can happen only where $f''(x)=0$.
    Setting $6(2x+3)=0$ yields $x=-\tfrac{3}{2}$. Checking the sign of $f''$ shows it is negative for $x<-\tfrac{3}{2}$ and pos
    itive for $x>-\tfrac{3}{2}$, confirming a sign change. The corresponding point on the curve is
    \[
        f\!\left(-\tfrac{3}{2}\right)=2\left(-\tfrac{3}{2}\right)^3+9\left(-\tfrac{3}{2}\right)^2+12\left(-\tfrac{3}{2}\right)+1=-\frac{27}{4}+\frac{81}{4}-18+1=-\frac{7}{2}.
    \]
    Therefore $f$ has a single inflection point at $\left(-\tfrac{3}{2},-\tfrac{7}{2}\right)$.}
    
\end{enumerate}

\newpage

\begin{enumerate}[label=(\alph*)]\addtocounter{enumi}{4} % Continue numbering from previous page
    \item (5 points) Sketch a graph of $y=f(x)$ and label its extrema, inflection points, and asymptotes.
    {\color{blue}\textbf{Discussion.} The cubic has no asymptotes, but we can summarize the qualitative behavior to guide a sket
    ch. As $x\to -\infty$, the leading term $2x^3$ drives $f(x)\to -\infty$; as $x\to\infty$, $f(x)\to\infty$. The graph passes
     through the local maximum $(-2,-3)$, then descends through the inflection point $\big(-\tfrac{3}{2},-\tfrac{7}{2}\big)$ whe
    re the concavity switches from down to up, continues to the local minimum $(-1,-4)$, and finally rises without bound. A corr
    ect sketch would show a smooth cubic curve reflecting these features, with concave down behavior to the left of $x=-\tfrac{3}
    {2}$ and concave up behavior to the right.}
    
\end{enumerate}

\newpage

% --- Question 4 ---
\section*{4. The spread of a rumor over time within a town with a population of 12000 people can be modeled by a logistic function}
\[ R(t) = \frac{12000}{1+e^{-t}}. \]
The logistic function $R(t)$ gives the number of people at time $t$ who have heard the rumor.

\begin{enumerate}[label=(\alph*)]
    \item (6 points) Find the linearization of $R(t)$ at $t=0$.
    {\color{blue}\textbf{Solution.} The linearization at $t=0$ is $L(t)=R(0)+R'(0)(t-0)$. We first compute
    \[
        R(0)=\frac{12000}{1+e^{0}}=\frac{12000}{2}=6000.
    \]
    Differentiating $R(t)=12000(1+e^{-t})^{-1}$ gives
    \[
        R'(t)=12000\cdot\frac{e^{-t}}{(1+e^{-t})^2}.
    \]
    Evaluating at $t=0$ yields $R'(0)=12000\cdot\dfrac{1}{(1+1)^2}=3000$. Therefore
    \[
        L(t)=6000+3000t.
    \]
    This line approximates the rumor spread for small values of $t$ near zero.}
    
    \item (3 points) Use linear approximation at $t=0$ to estimate the number of people in the town who have heard the rumor at time $t=1$.
    {\color{blue}\textbf{Solution.} Substitute $t=1$ into the linearization: $L(1)=6000+3000(1)=9000$. Thus, using the linear approximation, approximately $9{,}000$ residents have heard the rumor one unit of time after $t=0$.}

\end{enumerate}

\newpage

% --- Question 5 ---
\section*{5. Consider the curve $y=x^3+x$.}

\begin{enumerate}[label=(\alph*)]
    \item (5 points) Approximate the area under the curve between $x=0$ and $x=4$ by a Riemann sum of four rectangles using right endpoints.
    {\color{blue}\textbf{Solution.} Partition $[0,4]$ into four equal subintervals of width $\Delta x=1$. Using right endpoints gives sample points $x_i^*=1,2,3,4$. The Riemann sum is
    \[
        \sum_{i=1}^{4} f(x_i^*)\Delta x=\sum_{i=1}^4 (x_i^{*3}+x_i^*)\cdot 1=(1^3+1)+(2^3+2)+(3^3+3)+(4^3+4)=2+10+30+68=110.
    \]
    Thus the right-endpoint approximation is $110$ square units.}
    
    \item (4 points) Express the area under the curve as the limit of a Riemann sum.
    \textit{You do not need to evaluate the area for this part. You can leave your answer as a limit.}
    {\color{blue}\textbf{Solution.} Using $n$ subintervals of equal width $\Delta x=\frac{4}{n}$ and right endpoints $x_i^*=\frac{4i}{n}$, the area can be written as
    \[
        \lim_{n\to\infty}\sum_{i=1}^{n}\left[\left(\frac{4i}{n}\right)^3+\frac{4i}{n}\right]\cdot\frac{4}{n}.
    \]
    This limit is the definition of the definite integral of $x^3+x$ over $[0,4]$.}
    
\end{enumerate}

\newpage

\begin{enumerate}[label=(\alph*)]\addtocounter{enumi}{2} % Continue numbering from previous page
    \item (5 points) Find the exact area under the curve between $x=0$ and $x=4$.
    \textit{Hint: you do not have to use the limit from part (b); you find the area using any method covered in class.}
    {\color{blue}\textbf{Solution.} We evaluate the definite integral exactly:
    \[
        \int_0^4 (x^3+x)\,dx=\left[\frac{x^4}{4}+\frac{x^2}{2}\right]_0^4=\left(\frac{256}{4}+\frac{16}{2}\right)-0=64+8=72.
    \]
    Therefore the exact area is $72$ square units.}
    
    \item (2 points) Is the Riemann sum approximation of the area under the curve from part (a) an overestimate, an underestimate, or neither?
    {\color{blue}\textbf{Answer.} On $[0,4]$ the function $x^3+x$ is increasing, so using right endpoints makes each rectangle taller than the curve at the left edge. Consequently the right-endpoint Riemann sum overestimates the true area. This is consistent with comparing $110$ to the exact value $72$.}
    
\end{enumerate}

\newpage

% --- Question 6 ---
\section*{6. (6 points) Evaluate}
\[ \int \left( \frac{2}{\cos^2(t)} \sqrt{6 + \frac{\sin(t)}{\cos(t)}} + t \right) dt. \]
{\color{blue}\textbf{Solution.} Rewrite the integrand using trigonometric identities. Because $\dfrac{1}{\cos^2 t}=\sec^2 t$ and $\dfrac{\sin t}{\cos t}=\tan t$, the integral becomes
\[
    \int\left(2\sec^2 t\,\sqrt{6+\tan t}+t\right)dt.
\]
For the first term, let $u=6+\tan t$. Then $du=\sec^2 t\,dt$, so
\[
    \int 2\sec^2 t\,\sqrt{6+\tan t}\,dt=\int 2\sqrt{u}\,du=\frac{4}{3}u^{3/2}=\frac{4}{3}(6+\tan t)^{3/2}.
\]
The second term integrates directly to $\tfrac{t^2}{2}$. Combining the pieces and adding the constant of integration gives
\[
    \int \left( \frac{2}{\cos^2(t)} \sqrt{6 + \frac{\sin(t)}{\cos(t)}} + t \right) dt = \frac{4}{3}\left(6+\tan t\right)^{3/2}+\frac{t^2}{2}+C.
\]
This antiderivative differentiates back to the stated integrand, confirming the result.}

\end{document}
